\documentclass[a4paper,11pt]{kth-mag}
\usepackage[T1]{fontenc}
\usepackage{textcomp}
\usepackage{lmodern}
\usepackage[latin1]{inputenc}
\usepackage[swedish,english]{babel}
\usepackage{modifications}
\title{Design and implementation of a High Availability Controller}

\subtitle{Duis autem vel eum iruire dolor in hendrerit in
          vulputate velit esse molestie consequat, vel illum
          dolore eu feugiat null}
\foreigntitle{Lorem ipsum dolor sit amet, sed diam nonummy nibh eui
              mod tincidunt ut laoreet dol}
\author{Daniel Terranova}
\date{20140501}
\blurb{Master's Thesis at CSC\\
Supervisor: Tjoho\\
Examiner: Tjohej}
\trita{TRITA xxx yyyy-nn}
\begin{document}
\frontmatter
\pagestyle{empty}
\removepagenumbers
\maketitle
\selectlanguage{english}
\begin{abstract}
  Network devices and network management systems are mission critical.
  Therefore these kind of systems are often configured in redundant modes,
  one or several other systems can take over the function of a node which undergoes
  a system crash of some kind. This is often called High-Availability.

  The Tail-f System's Network Control System (NCS) is a network management
  framework. NCS offers centralised network configuration management
  functionality, along with providing options for extending the framework
  with additional features. The provisioning information managed by NCS are
  stored in its build in Configuration Database (CDB).

  The NCS product support high-availability through a replication
  architecture; one active master and a number of passive slaves.
  Slaves initialises the replication through a active connect to
  the master.

  The only thing NCS does is to replicate the CDB data amongst the members
  in the HA group. It does not perform any of the otherwise High-Availability
  related tasks such as running election protocols in order to elect a new
  master. This is the task of a High-Availability Controller (HAC) which
  must be in place and assume an external controller to take those decisions:
  switching master/slave nodes based on some criteria.

  At any point in time a slave can take over the role from the master.
  The decision is made by the controller which commands NCS to be
  master or slave by API calls.

  The background for this thesis work is to study the requirements and
  propose an implementation along with an evaluated prototype for a
  high-availability controller that could be a optional part of the
  Tail-f NCS product.

  Apart from developing the software High-Availability module, this work aims
  to summarise existing methods for design and to develop new methods
  for High-Availability controllers.

\end{abstract}
\clearpage
\begin{foreignabstract}{swedish}
  N�tverks element och system f�r �r konfigurering av n�tverk �r kritiska.
  Vilket inneb�r ofta att dessa system �r konfigurerade redundant.
  En eller flera system kan ta �ver en funktion fr�n ett system som genomg�r en
  system crash. Detta kallas oftast f�r h�g tillg�nglighet.

  Tail-f Systems NCS produkt �r ett n�tverk konfigurerings verktyg och ett ramverk. 
  NCS �r anv�ndbart som det �r, men kan �ven byggas ut av anv�ndaren
  med ytterligare funktioner. De enheter som hanteras av NCS lagras i 
  konfigurationsdatabasen (CDB).  
  
  NCS st�djer replikering mellan noder men det fins ingen automatisk mekanism som 
  avg�r vilken nod som avg�r vem ska vara den aktiv och passiv slav. Detta �r en uppgift
  �t en kontroller vilket m�ste finnas p� plats och som tar dessa beslut baserat p�
  vissa kriterier.
  
  I en given tidpukt kan en passiv slav ta �ver rollen fr�n den aktive noden.
  
  Bakgrunden f�r detta examens arbete �r att studera krav som st�lls p� en s�dan kontroller
  och f�resl� en implementation.
  
  F�rutom att utveckla programvara f�r enhetsidentifiering, syftar detta projekt till att sammanfatta 
  oika design och implementering befintliga metoder.
  
  \end{foreignabstract}
\clearpage
\tableofcontents*
\mainmatter
\pagestyle{newchap}
\chapter{Introduction}

Network management became a non trivial task, as networks grew and
incorporated different types of devices and configuration items.
Manual network management of large scale networks is unfeasible due to
the need for engineers specialized in different aspects and types of
network devices and their management, limited time, need to define a
strategy for configuration management, and the effort to track the
configuration state of large number of different devices.

These factors obviously increase the costs and effort required for network
management. To overcome these difficulties Network Management Systems (NMSs)
were developed. An NMS is a tool for network operators and engineers.
Such a tool enables centralized configuration management of many different
network devices, consolidates the storage of device configuration and state
information, pushes and pulls the configuration changes to and from the
devices. The state of the network must be persist to a storage where
it could be keep safe from system crash and failures that could occur.

Thus the configuration state information of the network is crusial for
the network to operate. Instead of having this information stored at
a single point in the system is made redundant.

Network Control System (NCS), an NMS developed by Tail-f Systems, is a
network management system and framework based on the
Network Configuration Protocol (NETCONF) and YANG
(a protocol that provides a unified interface to different devices and a
modelling language utilized by this protocol) supports the replication
mechanism between a master and a numer of passive slaves.

NCS support replication of the CDB configuration as well as of the
operational data kept in CDB. A group of NCS hosts consisting of a master,
and one or more slaves, is referred to as an HA group and sometimes as an
HA cluster.

A master has the primary role of a node in a HA cluster where modification
to the state of the network should be performed. Each passive slave
obtain the modifications done on the master node which means that
transaction that occurs on the master is replicated to the slaves.

When the master in some point in time faces a system crash the passive
slaves should be able to take the role of the master. A decision is
made by the controller which detect such failing situation and in turn
appoint some passive slave to take over the role of a master. The
remaning slaves should be notice of this situation by the controller in
turns receive commands of the new role situation.

\section{Problem Statement}

The only thing NCS does is to replicate the CDB data amongst the members in
the HA group. It doesn't perform any of the otherwise High-Availability
related tasks such as running election protocols in order to elect a new
master. This is the task of a High-Availability Framework controller which
must be in place. The HAFWC must instruct NCS which nodes are up and down using
API functions. Thus in order to use NCS configuration replication
we must first have a HAFWC in place.

A HAFWC must detect when nodes fail and instruct NCS accordingly.
If the master node fails, the HAFWC must elect one of the remaining slaves
and appoint it the new master. The remaining slaves must also be informed
by the HAFWC about the new master situation. NCS will never take any actions
regarding master/slave-ness by itself.

The HAFWC must also handle and detect a situation known as split brain which
indicates data inconsistencies originating from the maintenance of two
separate data sets with overlap in scope caused by network partition.

When a network partiton occurs the nodes in a HA cluster will not be able
to communicating with each other, believing that the other node is down will
be appointed a new master. At this point in time we end up with multiple
masters which could accept any modifications of the network state and failing
to synchronizing with each other lead to inconsistencies with each other.

A automatic reconciliation could not be performed without loosing
network state a manual intervene must be made to solve
the current situation.

The goal of this thesis work is to produce a prototype of High Availability
Controller (HAC) which is aware of the HA states of the HA Pairs. The
HAC should implement an internal state machine to keep track of the HA states.
This makes it possible to preserve database consistency.

theoretical model of how
a framework with two HA pairs should work and depict different events or
scenarios that could occur with expecting results base on
different criteria.

The model will be the base of a prototype which will be implemented
and tested.

\subsection{Restrictions and limitations}

This work is limited to only two nodes working as pairs. One master
and the other is a standby slave.

\part{Important Results}

\chapter{First One}

Aliquam et ante. Vivamus ultricies, neque eget iaculis interdum, lacus
quam hendrerit sapien, vel posuere justo nulla vitae arcu. Morbi
magna. Aliquam erat volutpat. Aenean mattis consequat nibh. Donec
lobortis sapien a enim. Cras mattis ultricies mi. Quisque venenatis.
Phasellus risus justo, vulputate non, tristique in, tristique vel,
neque. Phasellus pretium, dui nec dapibus laoreet, ligula enim laoreet
eros, tempus interdum massa turpis quis dolor. Etiam ultricies
condimentum neque. Maecenas pellentesque. Duis tortor. Aliquam ac
dolor. Vestibulum nisl. Nunc facilisis tincidunt mi. Morbi feugiat
velit vitae velit. Suspendisse potenti. Maecenas eget ante. Maecenas
blandit, urna at varius lacinia, lorem purus ullamcorper risus, non
pretium arcu libero at odio.

\section{Preliminaries}

Vestibulum dolor dolor, interdum eu, iaculis scelerisque, auctor at,
mi. Nullam ut erat quis sem bibendum dignissim. Vestibulum volutpat
nisl in tortor. Etiam pretium nulla et elit. Integer tortor. Ut metus
ligula, egestas non, ornare vel, bibendum eu, urna. Pellentesque
rutrum dui vel orci. Aliquam lacus lacus, varius a, sodales vitae,
egestas a, purus. Integer varius venenatis odio. Pellentesque dui
mauris, lacinia in, placerat lobortis, tempor ac, lectus.

\subsection{HA States}

Sed lobortis neque non mauris. Maecenas rhoncus tempor justo. Nunc vel
diam at dolor luctus tincidunt. Ut egestas. Aliquam eu turpis eu nisl
ultricies sollicitudin. Donec erat odio, fermentum id, malesuada id,
viverra vitae, lorem. Morbi scelerisque sagittis enim. Donec non ante
sed neque dictum consectetuer. Etiam bibendum odio quis est. Quisque
mollis magna et odio. Nunc nulla. Suspendisse magna felis, tincidunt
in, blandit non, tempus vel, orci.

\subsection{Definitions}

Sed laoreet tellus in massa. Fusce egestas dapibus wisi. Pellentesque
sit amet nulla. Donec dignissim rutrum urna. Integer et mi at urna
viverra vestibulum. Nullam justo tortor, vulputate nec, porta non,
consequat ac, lectus. Sed ligula. Nullam tristique. Nullam pulvinar.
Morbi at elit et lacus tristique nonummy. Vivamus nunc ante, imperdiet
at, luctus non, aliquam a, neque. In interdum sapien et wisi.
Curabitur ipsum justo, consequat eget, viverra eu, porta vel, lacus.
Suspendisse potenti. Proin wisi. In tristique neque at ipsum.
Phasellus ac justo eu nibh faucibus lobortis. Vestibulum lorem.

Duis blandit est id diam. Sed aliquet semper arcu. Curabitur sed eros
ut neque eleifend sagittis. Cras lacinia, nisl a sagittis scelerisque,
nibh risus varius pede, ac interdum lacus lorem quis quam. Mauris sit
amet orci a ligula porttitor nonummy. Aliquam malesuada turpis at diam
semper vestibulum. Phasellus ullamcorper. Aenean ultrices lacus a diam
lobortis posuere. Duis porttitor euismod turpis. Ut sem nulla, porta
vel, lacinia eget, rutrum non, justo. Phasellus risus sem, placerat
at, sodales ac, viverra vitae, magna. Fusce non magna. Mauris varius
vestibulum eros. Nullam nec quam in tortor ultrices varius. Nulla
tempus lectus ut tortor.

Ut vestibulum diam in lacus. Aenean vestibulum bibendum dolor.
Praesent in wisi. Aenean ornare faucibus orci. Donec mattis magna et
dui. Nullam vestibulum lobortis nibh. Etiam ultrices lorem nec est.
Duis ut nisl. Vivamus risus. Pellentesque habitant morbi tristique
senectus et netus et malesuada fames ac turpis egestas. Suspendisse
sagittis, dolor eget ullamcorper elementum, sem erat sagittis dolor,
sit amet tincidunt quam neque consectetuer quam. Maecenas at quam.
Etiam gravida. Sed nec enim. Fusce pulvinar. In pede metus, lobortis
a, ullamcorper quis, commodo vitae, turpis. Maecenas sed mi sit amet
odio suscipit ornare. Phasellus pede arcu, elementum sed, elementum
eu, viverra et, ligula. Aliquam metus nisl, convallis et, ultrices
vel, consequat quis, felis. Aenean gravida euismod urna.

\section{The Main Theorem}

Aliquam quis nibh quis justo elementum viverra. Vestibulum ipsum.
Integer sit amet urna id lorem condimentum pretium. Nam adipiscing
lobortis purus. Donec at libero id augue interdum vulputate. Curabitur
imperdiet suscipit metus. Curabitur ac quam sed lacus accumsan
posuere. Quisque pharetra mi sit amet enim. Curabitur quis elit. Lorem
ipsum dolor sit amet, consectetuer adipiscing elit. Sed ultricies
aliquet lorem. Proin posuere tincidunt diam. Donec quis orci non leo
elementum nonummy. Donec urna lectus, fringilla at, tempus id, auctor
a, wisi. Pellentesque habitant morbi tristique senectus et netus et
malesuada fames ac turpis egestas. Suspendisse blandit. Nam wisi.
Phasellus egestas lacus ut lorem. Suspendisse sapien. Fusce non dolor
ac odio tempus placerat.


\subsection{The Proof}

Suspendisse aliquam fringilla tortor. Nullam lacinia rutrum orci. Cras
pellentesque iaculis ligula. Fusce elit enim, nonummy ut, fermentum
sed, nonummy a, nunc. Suspendisse potenti. Nunc tortor dolor, eleifend
non, aliquam sit amet, vestibulum id, erat. Curabitur imperdiet.
Quisque in velit. Etiam a urna sed turpis scelerisque tempor. Praesent
in dolor eget massa congue aliquet. Praesent magna lacus, dictum a,
accumsan a, sollicitudin vel, nibh. Phasellus accumsan porta tortor.
Curabitur mauris. Quisque ut wisi. Sed aliquet molestie mi. Quisque ut
magna. Vivamus rhoncus urna in libero.

Nullam tristique tempus neque. Nunc ac tortor lobortis felis nonummy
lacinia. Pellentesque vestibulum facilisis quam. Aliquam aliquam,
lorem eget elementum euismod, urna dui rutrum libero, a imperdiet erat
augue ut ipsum. Morbi gravida turpis ut sapien. Aliquam ante sapien,
ultricies at, vulputate non, ullamcorper quis, dolor. Quisque velit
turpis, consectetuer vitae, facilisis feugiat, volutpat eget, nunc.
Quisque quis ipsum in diam elementum vulputate. Donec convallis sapien
ut tellus. Cras dictum turpis nec sem. Nunc augue. Aliquam nunc
tortor, semper vel, aliquam at, tempus ut, risus. Mauris feugiat nunc
id justo. Nunc ullamcorper magna vel urna. Morbi rutrum massa non
sapien. Fusce purus. Cras magna. In condimentum augue ac libero. Nulla
lacinia vulputate leo. Mauris lobortis molestie turpis.

Pellentesque neque odio, ornare a, hendrerit ac, vestibulum sit amet,
tortor. Vestibulum ante ipsum primis in faucibus orci luctus et
ultrices posuere cubilia Curae; Phasellus malesuada porta sem. Nunc
wisi neque, lobortis at, consequat sit amet, pulvinar eget, mauris.
Aliquam egestas, arcu ut tristique tempus, urna lorem pulvinar nisl,
at auctor lectus sem ut felis. Duis hendrerit, mi et varius venenatis,
nibh libero blandit mauris, bibendum pellentesque sapien dolor quis
velit. Vivamus nec augue ut velit eleifend accumsan. Fusce euismod.
Phasellus volutpat. Donec sollicitudin. Donec non mi eu lorem suscipit
tincidunt. Suspendisse vehicula. Integer iaculis diam ac quam.
Pellentesque velit mi, pulvinar eget, suscipit sed, rhoncus ut,
tortor. Etiam ultricies, wisi non ornare rutrum, ipsum nunc posuere
leo, vitae blandit est enim eu massa. Nulla sed felis vel nunc
eleifend venenatis. Sed non purus. Aliquam enim. Nulla neque massa,
elementum sit amet, rhoncus vel, rhoncus in, leo.

In convallis pellentesque quam. Fusce venenatis neque id justo.
Aliquam felis. Nullam vitae eros. Sed sit amet lorem. Vestibulum
elementum purus in nunc. Phasellus nec libero non ipsum ultrices
porttitor. Vestibulum ante ipsum primis in faucibus orci luctus et
ultrices posuere cubilia Curae; Praesent quis elit sed eros bibendum
tempor. Suspendisse tempor congue nisl. Nam a est. Aliquam erat
volutpat. Quisque tincidunt. Donec egestas dapibus diam. Aenean ante
mi, vulputate vitae, adipiscing in, fermentum luctus, wisi. In nibh
mauris, ultricies a, fermentum eget, faucibus pharetra, neque. Cras
consectetuer congue ipsum. Nulla eu metus. Sed at turpis.

\appendix
\addappheadtotoc
\chapter{RDF}\label{appA}

\begin{figure}[ht]
\begin{center}
And here is a figure
\caption{\small{Several statements describing the same resource.}}\label{RDF_4}
\end{center}
\end{figure}

that we refer to here: \ref{RDF_4}
\end{document}
